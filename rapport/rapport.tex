\documentclass{article}
\usepackage[utf8]{inputenc}
\usepackage[T1]{fontenc}
\usepackage[english]{babel}
\usepackage{hyperref}

\title{My Report}
\author{Your Name}
\date{\today}

\begin{document}

\maketitle

\newpage
\tableofcontents
\newpage

\section{Introduction}
Provide an overview of the internship, its context, and main objectives.

\section{Data Preparation}
The first step of the internship involved finding open source data on water quality in France. 
The internship supervisor provided several sources, notably the \textit{HubEau} (https://hubeau.eaufrance.fr/) 
and TODO website that offer water quality data by French communes.

It is important to note that the data can be provided in multiple ways:
\begin{itemize}
    \item \textbf{Databases}: These allow users to download the entire raw dataset for offline analysis.
    \item \textbf{APIs}: These provide access to portions of the data through URL-based queries, 
        enabling more targeted and up-to-date data. Especially used for online services using such data.
\end{itemize}

The HubEau platform is an API that serves as an interface to the underlying databases available on \texttt{data.gouv.fr}.
In this internship, we mostly focused on the databases, to explore and create tools of analysis first, 
and maybe later create services with the APIs.
\subsection{Databases}
Let's list the different databases available on the subject:

\begin{itemize}
    \item \textbf{Sandre Reference Databases} (\url{https://www.sandre.eaufrance.fr/}) 
        -- All datasets related to water quality in France are structured using standardized referentials defined by SANDRE. 
        These referentials provide a unified framework for parameters, codes, and nomenclatures, ensuring consistency 
        and interoperability across different data sources. The Sandre reference databases are therefore fundamental 
        for any analysis or integration of water quality data.
    \item \textbf{Sanitary Control of Distributed Water} (\url{https://www.data.gouv.fr/fr/datasets/resultats-du-controle-sanitaire-de-leau-distribuee-commune-par-commune/}, 
        \url{https://hubeau.eaufrance.fr/page/api-qualite-eau-potable}) 
        -- Contains results of sanitary controls of distributed water, with data available by municipality and updated 
        every few days to months. Parameters include microbiological, physical, and chemical indicators.
    \item \textbf{Sanitary Control of Tap Water} (\url{https://www.data.gouv.fr/fr/datasets/resultats-du-controle-sanitaire-de-leau-du-robinet/}) 
        -- Provides additional results on tap water quality.
    \item \textbf{Groundwater Quality} (\url{https://ades.eaufrance.fr/Recherche/Index/QualitometreAvance?g=9220e3}) 
        -- Provides groundwater quality measurements, though no API is available.
    \item \textbf{Quality of Groundwater and Surface Water} (\url{https://hubeau.eaufrance.fr/page/api-qualite-nappes}, 
        \url{https://hubeau.eaufrance.fr/page/api-qualite-cours-deau}) 
        -- APIs for accessing groundwater and surface water quality data from the ADES database.
    \item \textbf{Legislation on Drinking Water} (\url{https://www.legifrance.gouv.fr/loda/id/JORFTEXT000000274485/2022-11-28/}) 
        -- Provides criteria for drinking water.
\end{itemize}

Water quality is monitored at several key stages as it moves from natural sources to the consumer's tap. 
Analyses are first performed at the source—whether groundwater or surface water—to assess initial quality. 
After collection, water is treated at sanitary stations, where further analyses ensure it meets drinking standards 
before entering the distribution network. Finally, additional checks are conducted at random taps within 
the network to verify that quality is maintained up to the point of use.

For the purposes of this internship, we chose to focus our efforts on the \textbf{Sanitary Control of Distributed Water} database. 
This dataset offers comprehensive and regularly updated information on water quality at the municipal level, 
making it an ideal starting point for analysis and tool development.
Throughout this project, we will of course make extensive use of the SANDRE database itself. 
The SANDRE framework organizes water quality information by assigning a unique identifier, called \texttt{cdparametre}, 
to each measured element—whether it is a physical, chemical, or environmental parameter. In all the referenced databases, 
each data entry corresponds to the measurement of a single \texttt{cdparametre} at a specific point in time. 
This consistent structure enables unified analysis across datasets and simplifies the integration of new sources in the future.
\subsection{Data Cleaning}
The data cleaning phase was essential due to the heterogeneity and scale of the datasets. 
All data sources were provided in CSV format, but with differing delimiters and encoding schemes, 
necessitating careful standardization during import. In particular, the SANDRE reference database contained over one thousand columns, 
requiring a systematic approach to column selection.

To ensure analytical relevance and computational efficiency, only the columns pertinent to the study—such as parameter codes 
(\texttt{cdparametre}), measurement values, temporal information, and geographical identifiers—were retained. 
This selective extraction significantly reduced dataset complexity and facilitated subsequent processing.

Given the overall database size of approximately 15GB, it was not feasible to load the entire dataset into memory. To address this, 
the data was partitioned by parameter, resulting in separate files for each \texttt{cdparametre}. 
This strategy enabled parameter-wise processing, optimizing memory usage and allowing for scalable data manipulation.

Data integration from multiple sources was performed using shared keys, such as parameter codes and location identifiers, 
to ensure consistency and enable cross-referencing. Throughout the cleaning process, rigorous attention was paid to harmonizing data types 
and formats, as well as identifying and rectifying anomalies or inconsistencies. This comprehensive data preparation established 
a robust foundation for all subsequent analyses.

\section{Data Exploration}
Summarize the initial exploration of the dataset, with an emphasis on timeseries analysis.
\subsection{Sandre Parameters}
SANDRE provides an extensive list of parameters, each identified by a unique code (\texttt{cdparametre}).
We want to select the relevant parameters for our analysis, focusing on thoses that are most indicative of water quality.
It starts with the parameters considered for the french and european legislation, aswell as other parameters that we might consider important.
We also want to select the parameters with a sufficient sample size, both-timewise, and geographically-wise.

Parameters fall into two main types: those with continuous values for every measurement (e.g., temperature), 
and detection parameters, which are usually zero but occasionally indicate the presence of a specific substance or organism.
Theses 2 types are important to distinguish, as they require different analysis methods. For now we will only focus on the continuous parameters.



So, we decided on 4 filters to determine the parameters to keep:
- \textbf{Sample Size}: The average number of measurements of the parameter for each commune.
- \textbf{Geographical Coverage}: The number of communes where the parameter is measured.
- \textbf{Numerical Continuity}: The percentage of non-zero measurements of the parameter.

IMAGE THRESHOLD
In order to select the thresholds, an overview of the distribution of theses 3 parameters was made.
We selected the following thresholds values:
\begin{itemize}
    \item \textbf{Sample Size}: 100 measurements per commune
    \item \textbf{Geographical Coverage}: 50 communes
    \item \textbf{Legislative Relevance}: 10% non-zero measurements
\end{itemize}

We obtained the following parametesr, of which, only X are present in the legislation:
\begin{itemize}
    \item Température de l'Eau
    \item Potentiel en Hydrogène (pH)
    \item Conductivité à 25°C
    \item Hydrogénocarbonates
    \item Chlorures
    \item Sulfates
    \item Nitrates
    \item Dureté totale
    \item Titre alcalimétrique complet (T.A.C.)
    \item Potassium
    \item Magnésium
    \item Calcium
    \item Sodium
    \item Chlore libre
    \item Chlore total
    \item Carbone Organique
    \item Equilibre calcocarbonique de l’eau destinée à la consommation humaine
    \item Nitrates/50 + Nitrites/3
    \item Température de mesure du pH
    \item pH d'equilibre
    \item Fluorure anion
\end{itemize}

\newpage
\section{Analysis Methods}
We performed a compre3hensive analysis of the water quality data, focusing on both geographical and temporal aspects.

\subsection{Temporal Analysis}

First and foremost, for any temporal analysis, we first need a preprocessing of the data, to de-regionalize the measurements, allowing for a broader analysis of the parameters.
This involves several steps for each parameter:
\begin{itemize}
    \item Center-reducing all measurements by the commune mean and standard deviation.
    \item Aggregating The data.
    \item Removing outliers, defined as values outside 3 standard deviations from the centered reduced mean (0), and standard deviation (1).
    \item Smoothing the data using a 60-day rolling window to reduce noise.
    \item Re-adding the global mean and standard deviation to the data.
\end{itemize}

Because we need the mean and standard deviation by commune, this implies a decent sample size for each communes.
Thus we filter out the communes with less than X measurements for each parameter.
We can now perform a series of timeseries analyses, including:
\begin{itemize}
    \item Linear regression to identify trends over time.
    \item Fourier analysis to detect seasonal patterns and periodicities in the data.
\end{itemize}

From this, we can identify the seasonal parameters, such as temperature, or more indirectly, the pH.

\subsection{Correlation Analysis}
Correlation analysis for timeseries aims to identify relationships between different parameters, 
particularly over time. For time series data, we compute Pearson correlations, which can detect correlations with a lag.
However, strong correlations can arise simply due to shared seasonality or trends. 
To obtain meaningful results, it is essential to detrend and deseasonalize 
the data before calculating correlations, ensuring that observed relationships 
reflect genuine interactions rather than coincident temporal patterns.
We also need a centered reduced and de-regionalized data, as described in the previous section.

Note that making a correlation matrix requires a significant amount of memory,
since we have to process all possible pairs of parameters.
There was a significant effort toward optimizing the memory usage, as per the annex X.

\subsection{Potability Index}
The potability is only indicated by the respect of the legislation, which defines the maximum allowed values for each parameter.
So, by default, it follows a binary classification, whether it is potable or not.
We aimed at creating a continuous value, that would indicate the potability of the water with more information.
To do that, for continuous numerical values, we calculated a percentage of respect of the maximum allowed values for each parameter.
For the detection parameters, we calculated the percentage of measurements that were above the maximum allowed value.
And finally, we took the maximum of theses percentage across all parameters.
Note that this is a very simplistic approach, and does not take into account the importance of each parameter, aswell as their scale.

\section{Geographical Analysis}


\section{Results}
At first, we limited our analysis to the continuous parameters.


\section{Future Work}
Suggest possible directions for future research and improvements.

\section{Conclusion}
Summarize the key findings and contributions of the internship.

\appendix
\section{Annex}

\subsection{List of Acronyms}
\begin{itemize}
    \item SANDRE: Service d’Administration Nationale des Données et Référentiels sur l’Eau
    \item API: Application Programming Interface
    \item CSV: Comma-Separated Values
    \item T.A.C.: Titre Alcalimétrique Complet
\end{itemize}

\subsection{Example of Data Structure}
\begin{verbatim}
| Date       | Commune   | cdparametre | Value | Unit  |
|------------|-----------|-------------|-------|-------|
| 2023-01-01 | Paris     | 1234        | 7.2   | pH    |
| 2023-01-01 | Lyon      | 5678        | 0.05  | mg/L  |
\end{verbatim}

\subsection{Useful Links}
\begin{itemize}
    \item \url{https://hubeau.eaufrance.fr/}
    \item \url{https://www.sandre.eaufrance.fr/}
    \item \url{https://www.data.gouv.fr/}
    \item \url{https://eur-lex.europa.eu/FR/legal-content/summary/drinking-water-essential-quality-standards.html}
\end{itemize}

\subsection{Summary Table: French vs European Criteria}
\begin{table}[h!]
\centering
\begin{tabular}{|l|l|l|l|}
\hline
\textbf{Code} & \textbf{Substance} & \textbf{French Limit} & \textbf{European Limit} \\
\hline
1376 6063 & Antimoine & 5,0 µg/l & 10 µg/l \\
1369 & Arsenic total & 10 µg/l & 10 µg/l \\
1396 & Baryum & 1,0 mg/l & -- \\
1362 1818 & Bore & inf mg/l & 1,5 mg/l \\
1388 & Cadmium & 3,0 µg/l & 5,0 µg/l \\
1389 7976 1371 & Chrome & 50 µg/l & 25 µg/l \\
1392 & Cuivre & 1,0 mg/l & 2,0 mg/l \\
1084 1390 & Cyanures & 70 µg/l & 50 µg/l \\
7073 & Fluorures & 5,0 mg/l & 1,5 mg/l \\
1382 5253 6051 3358 1928 & Plomb & 10 µg/l & 5 µg/l \\
1394 6074 & Manganèse & 500 µg/l & -- \\
1387 & Mercure & 1,0 µg/l & 1,0 µg/l \\
1386 & Nickel & 20 µg/l & 20 µg/l \\
1340 & Nitrates & 50 mg/l & 50 mg/l \\
1339 & Nitrites & 0,1 mg/l & 0,50 mg/l \\
1385 & Sélénium & 10 µg/l & 20 µg/l \\
1457 & Acrylamide & -- & 0,10 µg/l \\
1114 & Benzène & -- & 1,0 µg/l \\
1115 & Benzo(a)pyrène & -- & 0,010 µg/l \\
2766 & Bisphénol A & -- & 2,5 µg/l \\
1751 & Bromates & -- & 10 µg/l \\
1752 & Chlorates & -- & 0,25 mg/l \\
1735 & Chlorites & -- & 0,25 mg/l \\
1161 & 1,2-dichloroéthane & -- & 3,0 µg/l \\
X & Épichlorhydrine & -- & 0,10 µg/l \\
9064 & Acides haloacétiques (AHA) & -- & 60 µg/l \\
2058 & Microcystine-LR & -- & 1,0 µg/l \\
X & Pesticides & -- & 0,10 µg/l \\
6276 6277 & Total pesticides & -- & 0,50 µg/l \\
X & Total PFAS & -- & 0,50 µg/l \\
8847 9268 & Somme PFAS & -- & 0,10 µg/l \\
X & Hydrocarbures aromatiques polycycliques & -- & 0,10 µg/l \\
2963 1272 1286 & Tétrachloroéthylène et trichloroéthylène & -- & 10 µg/l \\
2036 & Total trihalométhanes & -- & 100 µg/l \\
1361 & Uranium & -- & 30 µg/l \\
1753 & Chlorure de vinyle & -- & 0,50 µg/l \\
\hline
\end{tabular}
\caption{Summary Table: French vs European Criteria for Drinking Water}
\end{table}

\end{document}